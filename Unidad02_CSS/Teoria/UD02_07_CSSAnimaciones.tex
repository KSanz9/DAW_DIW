% Created 2019-11-14 jue 16:16
\documentclass[11pt]{article}
\usepackage[utf8]{inputenc}
\usepackage[T1]{fontenc}
\usepackage{fixltx2e}
\usepackage{graphicx}
\usepackage{longtable}
\usepackage{float}
\usepackage{wrapfig}
\usepackage{rotating}
\usepackage[normalem]{ulem}
\usepackage{amsmath}
\usepackage{textcomp}
\usepackage{marvosym}
\usepackage{wasysym}
\usepackage{amssymb}
\usepackage{hyperref}
\tolerance=1000
\usepackage[newfloat]{minted}
\hypersetup{colorlinks=true,linkcolor=black}
\author{Angel Berlanas}
\date{\today}
\title{UD02 - Animaciones CSS}
\hypersetup{
  pdfkeywords={},
  pdfsubject={},
  pdfcreator={Emacs 25.2.2 (Org mode 8.2.10)}}
\begin{document}

\maketitle
\tableofcontents


\section{Animaciones CSS}
\label{sec-1}

\subsection{¿Qué son las Animaciones CSS?}
\label{sec-1-1}

Las animaciones CSS3 permiten animar la transición entre un estilo CSS y
otro, son muy similares a las \verb~transiciones~ que vimos en clases anteriores. 

Las animaciones constan de dos componentes: un estilo que describe la
animación CSS y un conjunto de fotogramas que indican su estado inicial y final,
así como posibles puntos intermedios en la misma.

\subsection{Propiedades}
\label{sec-1-2}

A continuación vamos a ver las diferentes propiedades que pueden tener las
\textbf{Animaciones CSS}, como vereís se trata de conceptos muy similares a los que
están descritos en los apartados anteriores.

\subsubsection{animation-name}
\label{sec-1-2-1}

Se trata de la propiedad que establece el nombe de la regla (\verb~@keyframes~) que
describe los fotogramas de la aplicación.

Pueden especificarse más de uno, en ese caso irán separados por \textasciitilde{},\textasciitilde{}.

\subsubsection{animation-duration}
\label{sec-1-2-2}

Especificado en \emph{segundos} lo que dura la animación \textbf{completa}.

\subsubsection{animation-iteration-count}
\label{sec-1-2-3}

Indica cuantas veces debe realizarse la animación.

Los valores posibles son:

\begin{itemize}
\item \verb~infinite~
\item \verb~<number>~
\item $x^2x$
\end{itemize}

Donde \verb~<number>~ establece un número de iteraciones, se pueden ponder decimales,
y se aplicará ese porcentaje de la animación.

\begin{minted}[]{css}
animation-iteration-count: 2.3;
\end{minted}

Realizará la animación 2 veces completas y \textbf{un tercio} de la misma.

\subsubsection{animation-timing-function}
\label{sec-1-2-4}

Se trata del mismo concepto que vimos en las transiciones, \emph{cómo cambia}
\emph{en el tiempo} la propia animación, acelerándose o frenándose según se
especifique.

Este parámetro puede ser establecido en cada uno de los \verb~@keyframes~ que tenga
la animación.

Y aquí también podemos usar \verb~cubic-bezier~ como posibles valores:

\begin{center}
\begin{tabular}{l}
Valores\\
\hline
ease-in\\
ease-out\\
ease-in-out\\
linear\\
cubic-bezier\\
\end{tabular}
\end{center}


\subsubsection{@keyframes}
\label{sec-1-2-5}

\verb~@keyframes~ es la parte en la que programamos la animación, indicando mediante
una serie de palabras \emph{clave} y los porcentajes de la animación.

\begin{minted}[]{css}
@keyframes pixar {

   from {
	top:100px;
	animation-timing-function:ease-in;
   }
   50%{
       top:125px;
       animation-timing-function:ease-out;
   }
   to {
	top:150px;
   }
}
\end{minted}

\subsubsection{Un ejemplo completo}
\label{sec-1-2-6}

\newpage
\begin{minted}[]{css}
.elemento {
    width: 142px;
    height: 142px;
    background-color: #00AA66;
    position: relative;
    margin: 0 auto;
    animation-name: saltarin;
    animation-duration: 2s;
    animation-iteration-count: infinite;
}

@keyframes saltarin {
    from {
	top: 150px;
	animation-timing-function: ease-out;
    }
    25% {
	top: 50px;
	animation-timing-function: ease-in;
    }
    50% {
	top: 200px;
	animation-timing-function: ease-out;
    }
    75% {
	top: 75px;
	animation-timing-function: ease-in;
    }
    to {
	top: 150px;
    }
}
\end{minted}


\subsubsection{Ejercicio 13 : ¡Estan vivos!¡Vivooooos!}
\label{sec-1-2-7}

La investigación de los pergaminos ha concluido con éxito. Los \emph{Seres de
Sedefkar} están despertando y ya son capaces de realizar pequeñas tareas.

Utilizando dos \emph{papiros} que se ha encontrado en la Biblioteca de la Universidad
de Miskatonic, que están escritos en \verb~JS~ y \verb~CSS~. Los investigadores deben
recomponer los fragmentos perdidos que permitirán indicar a los diferentes \emph{Seres de}
\emph{Sedefkar} cuál será su tarea, ya que si no los controlarámos los \emph{Seres de
Sedefkar} se unirán bajo la voluntad de Cthulhu y acabarán con la cordura de los
investigadores.

Sobre los pergaminos anteriores añadiremos dos áreas, que al hacer click sobre
ellas, nos permitirán indicarle la acción sobre los \emph{Seres de Sedefkar}.

\begin{itemize}
\item Se estableceran dos \emph{glifos de órdenes} en la parte superior del tablero, al
lado del botón que hacer surgir las cajas. El primero hará girar al \emph{Ser de
Sedefkar} y el segundo hará un recorrido en V y volverá al origen.
\item Si el \emph{Ser de Sedefkar} está visible ha de hacer caso, en caso contrario, la
orden se pierde.
\item Si el \emph{Ser de Sedefkar} ya está ejecutando una orden, debe dejar de
ejecutarla y pasar a realizar la nueva orden.
\item Si se hace click fuera de un \emph{Ser de Sedefkar} la orden también se pierde.
\item (\textbf{Opcional}) Establecer una tercera orden que hace el \emph{Ser de Sedefkar} deje
de hacer \emph{cosas}.
\end{itemize}
% Emacs 25.2.2 (Org mode 8.2.10)
\end{document}

% Created 2019-11-18 lun 10:40
\documentclass[11pt]{article}
\usepackage[utf8]{inputenc}
\usepackage[T1]{fontenc}
\usepackage{fixltx2e}
\usepackage{graphicx}
\usepackage{longtable}
\usepackage{float}
\usepackage{wrapfig}
\usepackage{rotating}
\usepackage[normalem]{ulem}
\usepackage{amsmath}
\usepackage{textcomp}
\usepackage{marvosym}
\usepackage{wasysym}
\usepackage{amssymb}
\usepackage{hyperref}
\tolerance=1000
\usepackage[newfloat]{minted}
\hypersetup{colorlinks=true,linkcolor=black}
\author{Angel Berlanas}
\date{\today}
\title{UD03 - Migración a PostgreSQL}
\hypersetup{
  pdfkeywords={},
  pdfsubject={},
  pdfcreator={Emacs 25.2.2 (Org mode 8.2.10)}}
\begin{document}

\maketitle
\tableofcontents


\section{Introducción}
\label{sec-1}

Después de ver todos los pasos que se han seguido para conseguir 
realizar la migración desde un servicio basado en MongoDB a PostgreSQL, el
\emph{director} del \emph{Heraldo de Miskatonic}, el periódico de la Universidad de
Miskatonic, ha decidido migrar su periódico para no quedarse \emph{atras} de la
modernidad.

El \emph{director} ha llegado al equipo de desarrolladores y les ha encargado la
enorme tarea de realizar la migración, les ha dado 1 semana, ya que considera
que si los del \emph{periodicucho} The Guardian lo han hecho en unos meses, a sus
\emph{developers} no les puede costar más de una semana.

\section{Plan de acción}
\label{sec-2}

Los desarrolladores no sabían que hacer\ldots{}hasta que se les ha ocurrido que van
a darle al \emph{director} \textbf{exactamente} lo que quiere, una interfaz con un único
botón que al pulsarlo el directo \emph{migrará} los datos y todo funcionará a la
perfección.

La única esperanza es preparar en \emph{una semana} esta interfaz. Sin embargo hay
una serie de reglas que no podemos romper.

\section{Restricciones}
\label{sec-3}

No sabemos todavía cuantos pasos tendrá la migración, ya que el \emph{director} va
cambiando de parecer a cada rato, lo único en lo que nos podemos basar es que
cada uno de esos pasos tendrá un \verb~data-step~ que nos indicará en qué orden se
ejecutará.

No podemos tocar el código \verb~html~ que nos suministran, así como debemos
respetar los métodos que aparecen escritos en el fichero \verb~js~ y \verb~css~.

\begin{minted}[]{javascript}
function startMigration(){

    // Fragmentos perdidos
    // ^(;,;)^
}
\end{minted}

¡Mucha suerte a tod@s!
% Emacs 25.2.2 (Org mode 8.2.10)
\end{document}
